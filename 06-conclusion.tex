\section{まとめ}

% 本研究では, 地理的分散環境下に適した RW 実行エンジンを提案した. 提案手法では, 自律的 RW 実行という本手法の用途を意識することに加え, RWer 単位での独立性という RW 演算の特徴に注目し, UDP 通信を使用した非同期処理を採用した. UDP 通信を利用することで, WAN におけるスループットの低下を抑えられる. そして非同期処理を採用することで, 各データセンターによる自律的な RW 実行を実現することができる. また, 提案手法は過去に終了した RWer の経路を再利用することで, RW 遷移時の通信量を削減する機能を有する. 

% 地理的分散環境下を想定した実験の結果, 提案手法は既存手法に比べ $1.6 \sim 7.1$ 倍高速であることがわかった. また提案手法は RTT, パケットロス率が高ければ高いほど, さらにグラフ分割精度が悪ければ悪いほど有効であることがわかった. さらにサーバ数を変化させる実験により, 提案手法が既存手法に比べてシステム規模に対するスケーラビリティに優れていることがわかった. 

本研究では, 地理的分散環境下に適した RW 実行エンジンを提案した. 本研究は, 地理的分散環境下において各データセンターが, 自身が保有する頂点を始点とした RWer の生成・処理を自律的に行い, 結果をそのデータセンターの所在地域が活用することを想定する. この自律的な RW 実行を既存手法の同期処理で実現しようとする場合, 各データセンターにおけるアプリケーション内の RW 実行ごとに全データセンターで同期を行う必要がある. 地理的分散環境下においては, データセンターの増加により同期のための通信が増えることに加え, グラフ分割精度の悪化により同期の回数も増える. そして低帯域・高遅延の通信環境 (WAN) ではこの同期のために何度も発生する通信のオーバヘッドがさらに増加する. 対して提案手法は, 自律的 RW 実行という本手法の用途を意識することに加え, RWer 単位での独立性という RW 演算の特徴に注目し, UDP 通信を使用した非同期処理を採用した. UDP 通信を利用することで, WAN における性能低下を抑えられる. そして非同期処理を採用することで高コストな同期通信を省略し, 各データセンターによる自律的な RW 実行を実現することができる. また, 提案手法は過去に終了した RWer の経路を再利用することで, RW 遷移時の RWer の送信を一部スキップし, 通信量を削減した. 

地理的分散環境を想定した実験の結果, 提案手法は既存手法に比べ 1.6 (経路再利用なし) $\sim$ 7.1 (経路再利用あり) 倍高速であることがわかった. また提案手法は既存手法と異なり RTT, パケットロス率が大きくグラフ分割精度が悪いほど有効であり, サーバ台数を増やした実験で性能が悪化しなかったことからシステム規模に対するスケーラビリティにも優れていることがわかった. 

\section{今後の課題}

\subsection{データセンター内処理を意識したシステム}

提案手法では, 簡単のためデータセンターを一つのサーバとみなして実装を行った. しかし実際はデータセンターは複数のサーバで構成されているため, そのデータセンター内のサーバクラスタでの処理も考える必要がある. 例えば, データセンター内のサーバクラスタでの RW 実行では既存手法のような同期処理を行い, データセンターを一つの塊と見たときの全体としての RW 実行では提案手法のように非同期処理を行うといったアイデアが考えられる. 

\subsection{RWer の経路情報の保存方法}

本研究においては, 終了した RWer を生成サーバに送信して, その生成サーバのみが RWer の経路情報を保存していたが, 他にもいくつか方法が考えられる. 例えば, 終了した RWer を生成サーバだけでなく経由サーバ全てに送信する方法である. これは RWer の経路情報を確認すれば実現可能である. このようにすることで効率的に RWer の経路情報を保存することができる. また, 生存している RWer が他のサーバへ送信されるタイミングでそのサーバに現時点での RWer の経路情報を保存するといった手法も考えられる. 