\begin{center}
\Huge{論文要旨}
\end{center}

% Random Walk (RW) はグラフ解析技術として注目されている. 
% 一方, 解析対象となるグラフデータは近年大規模化が進み, WAN における転送コストの制約やプライバシ規制のため, 
% 一箇所に集中させず地理的に分散した環境下で保持されることが珍しくない. 
% しかしながら, このような WAN 環境では RTT やパケットロス率が増加するだけでなく, 
% グラフの全体把握が必要なグラフ分割も適用が困難となる. 
% 結果として主に単一データセンター内に保存されたグラフデータを対象とする既存 RW 手法を WAN 環境に適用すると
% 演算性能が大きく劣化する. 

% そこで本研究では, 地理的分散環境下であっても性能を維持することができる分散グラフ RW エンジンを提案する. 
% 提案手法は RW 演算の独立性に注目し, UDP を使用した非同期処理を採用した. 
% さらに Random Walker の経路再利用により通信量を削減する. 

% 地理的分散環境下を想定した実験の結果, 提案手法は既存手法に比べ $1.6 \sim 7.1$ 倍高速であることを明らかにした. 
% また提案手法は RTT, パケットロス率が大きくグラフ分割精度が悪いほど有効であり, 
% システム規模に対するスケーラビリティにも優れていることを明らかにした. 

% Random Walk (RW) はグラフ解析技術として注目されている. 一方, 解析対象となるグラフデータは近年大規模化が進み, WAN における転送コストの制約やプライバシ規制のため, 一箇所に集中させず地理的に分散した環境下で保持されることが珍しくない. しかしながら, このような WAN 環境では RTT やパケットロス率が増加するだけでなく, グラフの全体把握が必要なグラフ分割も適用が困難となる. 結果として主に単一データセンター内に保存されたグラフデータを対象とする既存 RW 手法を WAN 環境に適用すると演算性能が大きく劣化する. 

% そこで本研究では, 地理的分散環境下であっても性能を維持することができる分散グラフ RW エンジンを提案する. 提案手法は RW 演算が Random Walker (RWer) 単位で独立していることに注目し, UDP を使用した非同期処理を採用した. さらに RWer の経路再利用により RW 遷移時の通信量を削減する. 

% 地理的分散環境下を想定した実験の結果, 提案手法は既存手法に比べ $1.6 \sim 7.1$ 倍高速であることを明らかにした. また提案手法は RTT, パケットロス率が大きくグラフ分割精度が悪いほど有効であり, システム規模に対するスケーラビリティにも優れていることを明らかにした. 

Random Walk (RW) はグラフ解析において広く利用されている. 解析対象となるグラフデータは世界中のデータセンター (DC) に保存されており, 世界規模で展開される検索サービスの運用のためには, その世界中に分散したデータを解析する必要がある. このようなグラフ解析をするとき, 中央集権的に全ての部分グラフを一箇所に集めてから解析することが理想だが, グラフの大規模化やグラフの変化により WAN での通信コストが非常に高くなることや, プライバシー規制によって, これは実現不可能である. そのため, 地理的に分散した環境下における分散グラフ RW 実行エンジンの開発が重要となる. 
また, 将来グラフがより大規模化していくことを考慮すると, 全世界に分散するグラフ上の全頂点から RW を実行することは現実的ではなく, 興味がある一部頂点を始点とした RW の実行が主流になる. そのため本研究における地理的分散環境下での RW 実行は, 各 DC が, 自身が保有する頂点を始点とした Random Walker (RWer) の生成・処理を自律的に行い, 結果をその DC の所在地域が活用することを想定する. 
既存の分散グラフ RW 実行エンジンは単一の DC 内での実行を想定し TCP 通信を使用した Bulk synchronous parallel (BSP) モデルによる同期処理を行う. しかし本研究が想定する地理的分散環境下において, 低帯域・高遅延の WAN では TCP 通信の性能が大きく劣化することに加え DC の増加により同期のための通信量が増加する. また,  DC ごとのグラフ編集により全体として見たときのグラフ分割精度が悪化する場合, DC 間の RW 遷移が増えるため, DC 間での同期回数が増加する. そのため既存手法は本研究が想定する環境には適さない. 
そこで本研究では, 地理的分散環境下に適した分散グラフ RW 実行エンジンを提案する. 提案手法は, 地理的分散環境下における分散グラフ RW 実行エンジンの用途を意識することに加え, RWer 単位での独立性という RW 演算の特徴に注目し, UDP 通信を使用した非同期処理を採用した. UDP 通信を利用することで, WAN における性能低下を抑えられる. そして非同期処理を採用することで高コストな同期通信を省略し, 各 DC による自律的な RW 実行を実現することができる. また, 提案手法は過去に終了した RWer の経路を再利用することで, RW 遷移時の RWer の送信を一部スキップすることができる.
地理的分散環境を想定した実験の結果, 提案手法は既存手法に比べ 1.6 (経路再利用なし) $\sim$ 7.1 (経路再利用あり) 倍高速であることを明らかにした. また提案手法は既存手法と異なり RTT, パケットロス率が大きくグラフ分割精度が悪いほど有効であり, サーバ台数を増やした実験で性能が悪化しなかったことからシステム規模に対するスケーラビリティにも優れていることを明らかにした. 


\vspace{36pt}
\thispagestyle{empty}
\clearpage