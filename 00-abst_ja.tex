\begin{center}
\Huge{論文要旨}
\end{center}

% Random Walk (RW) はグラフ解析技術として注目されている. 
% 一方, 解析対象となるグラフデータは近年大規模化が進み, WAN における転送コストの制約やプライバシ規制のため, 
% 一箇所に集中させず地理的に分散した環境下で保持されることが珍しくない. 
% しかしながら, このような WAN 環境では RTT やパケットロス率が増加するだけでなく, 
% グラフの全体把握が必要なグラフ分割も適用が困難となる. 
% 結果として主に単一データセンター内に保存されたグラフデータを対象とする既存 RW 手法を WAN 環境に適用すると
% 演算性能が大きく劣化する. 

% そこで本研究では, 地理的分散環境下であっても性能を維持することができる分散グラフ RW エンジンを提案する. 
% 提案手法は RW 演算の独立性に注目し, UDP を使用した非同期処理を採用した. 
% さらに Random Walker の経路再利用により通信量を削減する. 

% 地理的分散環境下を想定した実験の結果, 提案手法は既存手法に比べ $1.6 \sim 7.1$ 倍高速であることを明らかにした. 
% また提案手法は RTT, パケットロス率が大きくグラフ分割精度が悪いほど有効であり, 
% システム規模に対するスケーラビリティにも優れていることを明らかにした. 

Random Walk (RW) はグラフ解析技術として注目されている. 一方, 解析対象となるグラフデータは近年大規模化が進み, WAN における転送コストの制約やプライバシ規制のため, 一箇所に集中させず地理的に分散した環境下で保持されることが珍しくない. しかしながら, このような WAN 環境では RTT やパケットロス率が増加するだけでなく, グラフの全体把握が必要なグラフ分割も適用が困難となる. 結果として主に単一データセンター内に保存されたグラフデータを対象とする既存 RW 手法を WAN 環境に適用すると演算性能が大きく劣化する. 

そこで本研究では, 地理的分散環境下であっても性能を維持することができる分散グラフ RW エンジンを提案する. 提案手法は RW 演算が Random Walker (RWer) 単位で独立していることに注目し, UDP を使用した非同期処理を採用した. さらに RWer の経路再利用により RW 遷移時の通信量を削減する. 

地理的分散環境下を想定した実験の結果, 提案手法は既存手法に比べ $1.6 \sim 7.1$ 倍高速であることを明らかにした. また提案手法は RTT, パケットロス率が大きくグラフ分割精度が悪いほど有効であり, システム規模に対するスケーラビリティにも優れていることを明らかにした. 

\vspace{36pt}
\thispagestyle{empty}
\clearpage