\begin{center}
  \Huge{Thesis Abstract}\\
  \vspace{10pt}
  \large{}
\end{center}


Random Walk (RW) is widely used in graph analysis. The graph data to be analyzed are stored in data centers (DC) around the world, and it is necessary to analyze such globally distributed data for the operation of global search services. When analyzing such graphs, it would be desirable to collect all subgraphs in one centralized location and then analyze them, but this is not feasible due to the extremely high cost of communication over WANs caused by large graphs and graph changes, as well as privacy regulations. Therefore, it is important to develop a distributed graph RW execution engine in a geographically distributed environment. 
Considering that graphs will become larger and larger in the future, it is not realistic to execute RW from all vertices on a globally distributed graph, and it will become mainstream to execute RW starting from some vertices of interest. Therefore, in the execution of RW in the geographically distributed environment in this research, each DC is assumed to autonomously generate and process a Random Walker (RWer) starting from its own vertex, and the results are assumed to be utilized by the region where the DC is located. 
The existing distributed graph RW execution engine assumes execution in a single DC and performs synchronous processing using the Bulk synchronous parallel (BSP) model with TCP communication. However, under the geographically distributed environment assumed in this research, the performance of TCP communication in a low-bandwidth, high-latency WAN is greatly degraded, and the amount of communication for synchronization increases due to the increase of DCs. In addition, when the accuracy of graph partitioning as a whole is deteriorated by editing the graph for each DC, the number of synchronization cycles among DCs increases due to the increase of RW transitions among DCs. Therefore, this existing method are not suitable for the environment envisioned in this research.
In this paper, we propose a distributed graph RW execution engine suitable for geographically distributed environments. The proposed method employs asynchronous processing using UDP communication, by paying attention to the feature of RW operations such as independence per RWer and being aware of the application of the proposed method. By using UDP communication, the performance degradation in WAN can be suppressed. And asynchronous processing eliminates the need for costly synchronous communication and enables autonomous RW execution by each data center. Moreover, the proposed method can skip some RWer transmissions during RW transitions by reusing the routes of RWers that have been terminated in the past.
Experimental results in a geographically distributed environment show that the proposed method is 1.6 (without route reuse) ~ 7.1 (with route reuse) times faster than the existing methods. Unlike existing methods, the proposed method is effective when the RTT and packet loss rate are large and the accuracy of graph partitioning is poor. The proposed method is also found to have excellent scalability with respect to the system size. This is because the performance of the proposed method does not deteriorate when the number of servers is increased.

\vspace{36pt}
\thispagestyle{empty}
\clearpage